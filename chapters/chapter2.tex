\section{Grundlagen des Machine Learning (3-4 Seiten)}
\subsection{Einordnung von Boosting im Kontext des maschinellen Lernens}
Boosting lässt sich dem überwachten Lernen (supervised Learning) zuordnen. Diese Hauptkategorie des maschinellen Lernens beschäftigt sich mit Modellen, die auf Datensätzen mit Ein- und Ausgabe-Paaren trainiert werden. Durch Training versucht das Modell anhand der Eingabe Werte die Ausgabe Werte zu schätzen und wird je der Abweichung zur tatsächlichen Ausgabe angepasst. Das Ziel ist es das Modell zu nutzen um Aussagen auf bisher unbekannten Daten zu treffen.
\newline
Ein Teilgebiet davon ist das Ensemble-Lernen, eine Strategie die mehrere Modelle kombiniert, um eine bessere Lösung zu finden als die eines Einzelmodells. Boosting ist neben Bagging eine bekannte Methode in diesem Bereich. Während Boosting meist durch iteratives trainieren versucht das Gesamtmodell zu verbessern, ist Bagging ein paralleler Prozess, der Versucht vor allem die Varianz zu reduzieren.


\subsection{Modern Approaches in Machine Learning}
Überblick über aktuelle Trends und Innovationen im Machine Learning
Vorstellung fortgeschrittener Techniken und Methoden
Diskussion über die Bedeutung von Deep Learning und künstlichen neuronalen Netzen
\subsection{Role of Boosting Algorithms in ML}
Einführung in Boosting-Algorithmen und ihre Relevanz
Spezifische Betrachtung von \gls{adaboost} und \gls{gradientboosting}
Vergleich von Boosting-Algorithmen mit anderen fortgeschrittenen Methoden
\subsection{Boosting Algorithms in Tabular Data Analysis}
Bedeutung von tabellenartigen Datensätzen in fortgeschrittenen ML-Anwendungen
Analyse, wie \gls{adaboost} und \gls{gradientboosting} bei tabellenartigen Daten effektiv sind
Fallstudien und Beispiele aus der Praxis, die den Einsatz dieser Algorithmen zeigen

Daten
Entscheidungsbäume
Traingsfehler
