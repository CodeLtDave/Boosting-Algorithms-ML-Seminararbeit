\section{Boosting}
Nach meiner eigenen Erfahrung als Student fällt oft auf, dass Gruppenarbeit nicht immer die erwarteten Vorteile bringt. Oft dominiert in Lerngruppen ein einzelner Student, der über fundiertes Wissen verfügt, die Diskussion und die Gruppenleistung spiegelt im Wesentlichen seine individuelle Leistung wider. 
\newline
In Fällen, wo alle Mitglieder einer Lerngruppe nur begrenztes Wissen zu einem Thema haben, kann die kollektive Leistung sogar hinter dem zurückbleiben, was man durch zufällige Antworten erwarten würde. Die Vorstellung, dass eine Gruppe von Individuen mit begrenztem Wissen gemeinsam Ergebnisse erzielen kann, die sowohl den Durchschnitt als auch jede individuelle Bestleistung übertreffen, widerspricht oft unserer Intuition. Selbst historische Weisheiten, wie sie in der Bibel gefunden werden, betonen die Risiken einer solchen Zusammenarbeit. Dort heißt es metaphorisch: „Wenn aber ein Blinder den andern führt, so fallen sie beide in die Grube“ (Mt 23,16; Mt 23,24; Lk 6,39; Röm 2,19).
\newline
\newline
Doch im Bereich des maschinellen Lernens offenbart sich ein ganz anderes Szenario. Hier ermöglicht das Boosting-Verfahren, dass die Kombination von schwachen Modellen zu einem leistungsstarken Gesamtsystem führt. Dieser Ansatz, der die aggregierte Intelligenz mehrerer einfacher Modelle nutzt, um komplexe Probleme zu lösen, steht im starken Gegensatz zu den oft enttäuschenden Ergebnissen menschlicher Gruppenarbeit mit begrenztem Wissen \cite[S.~3]{SchapireFreund2012}.

\subsection{Was ist Boosting am Beispiel Wettererkennung}
\begin{mdframed}
    \textbf{Boosting (ursprünglich Hypothesis Boosting)} bezeichnet eine beliebige Ensemble-Methode, bei der sich mehrere schwache Lerner zu einem starken Lerner kombinieren lassen \textcite[S.~191]{Geron2018}.
\end{mdframed}

Die genannte Definition des Boosting lässt sich gut anhand des Beispiels der Wettererkennung veranschaulichen. Betrachten wir folgende einfache Regeln (Schwache Lerner) zur Beurteilung, ob es regnet:

\begin{table}[h]
    \centering
    \begin{tabular}{|l|l|}
    \hline
    \textbf{Schwache Lerner}      & \textbf{Grenzwert}   \\ \hline
    Nasser Boden               & Ja                  \\ \hline
    Wolken am Himmel           & Ja                  \\ \hline
    Hohe Luftfeuchtigkeit      & $>$ 80\%             \\ \hline
    Personen mit Regenschirm   & Ja                  \\ \hline
    Außentemperatur            & $>$ 0°C             \\ \hline
    \end{tabular}
    \caption{Individuelle Vorhersagen der Schwache Lerner}
    \label{tab:weak_learners}
\end{table}

Beispielsweise ist ein nasser Boden zwar eine Voraussetzung und ein guter erster Filter, allerdings könnte der Boden genauso gut durch einen Rasensprenger nass sein.
\newline
Die Temperatur ist hingegen ein relativ schlechtes Indiz für die Frage, ob es gerade regnet. Es unterscheidet aber den Fall Regen und Schnee und ist somit trotzdem essentiell für die Klassifikation.

\subsubsection{Erzeugung eines starken Lerners}
Der nächste Schritt ist es, die Aussagen der schwachen Lerner in ein nützliches Model zusammenzufassen, einen sog. `starken Lerner'. Im einfachsten Fall lässt man die schwachen Lerner mit gleicher Stimmkraft abstimmen. Aus dem Stimmverhältnis der Vorhersagen der schwachen Lerner lässt sich in unserem Fall die Regenwahrscheinlichkeit ableiten.

\begin{table}[h]
    \centering
    \begin{tabular}{|l|l|l|l|l|l|}
    \hline
    Nasser Boden & Wolken & Hohe Luftfeuchte & Regenschirm & Über 0°C & Regenwahrscheinlichkeit (\%) \\ \hline
    Ja & Ja & Ja & Ja & Ja & 100\% \\ \hline
    Ja & Ja & Nein & Nein & Ja & 60\% \\ \hline
    Nein & Ja & Ja & Ja & Ja & 80\% \\ \hline
    Ja & Nein & Ja & Nein & Ja & 60\% \\ \hline
    Nein & Nein & Nein & Nein & Ja & 20\% \\ \hline
    Ja & Ja & Ja & Nein & Nein & 60\% \\ \hline
    Nein & Ja & Nein & Ja & Ja & 80\% \\ \hline
    ... & ... & ... & ... & ... & ... \\ \hline
    \end{tabular}
    \caption{Wahrheitstabelle zur Vorhersage von Regen basierend auf schwachen Lernern. Die Werte sind frei gewählt und haben nicht notwendigerweise einen echten Bezug zur Realität, sondern dienen der Veranschaulichung.}
    \label{tab:rain_prediction}
\end{table}

\subsubsection{Gewichtete Abstimmung}
Im Anwendungsfall würde man sich nicht mit einer ungewichteten Abstimmung zufrieden geben. 

\subsection{Schwaches Lernen Annahme}
Voraussetzung hierfür ist, dass SEite 4 