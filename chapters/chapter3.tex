\section{Boosting}
Aus meiner eigenen Erfahrung weiß ich, dass es selten besonders positive Effekte hat, zusammen mit Kommilitonen eine Klausur zu lösen. Im Regelfall wird es einen Studenten geben, der sich relativ zur Gruppe, am besten auskennt und durch sein überdurchschnittliches Engagement, die Gruppenleistung auf etwa das Niveau bringt, was seiner Eigenleistung entsprochen hätte. 
\newline
In einem extremeren Fall, bei dem alle Studenten sich wenig Wissen zum Thema angeeignet haben, dass ihre Antworten nur geringfügig besser sind als zufälliges Raten, gibt es die realistische Chance, dass die Gruppenleistung sogar schlechter ausfällt, als hätte man tatsächlich geraten.
\newline
\newline
Die Vorstellung, dass durch Zusammenarbeit von Personen mit wenig Wissen Ergebnisse erzielt werden, die weit über dem Durchschnittleistung, als auch der individuellen Bestleistung liegen, scheint für Menschen sehr unrealistisch. Sogar die Bibel warnt metaphorisch vor der Zusammenarbeit zweier für eine Aufgabenstellung nicht geeigneter Personen. Es heißt: `Wenn aber ein Blinder den andern führt, so fallen sie beide in die Grube.' (Mt 23,16; Mt 23,24; Lk 6,39; Röm 2,19). Umso mehr überraschender scheint es, dass gerade im Bereich des Machine Learning die Zusammenarbeit schwacher Modelle zu erstaunlich starken Modellen kombiniert werden können.

\subsection{Was ist Boosting}