\section{Einleitung}

In der dynamischen Welt des maschinellen Lernens (ML) haben sich Boosting-Algorithmen als revolutionär erwiesen, indem sie beeindruckende Leistungen in einer Vielzahl von Anwendungen erbringen \parencite[Kapitel 1.1.2]{SchapireFreund2012}. Besonders bei tabellarischen Datensätzen erzielen sie in fast allen Bereichen die besten Ergebnisse. 
\newline
In dieser Seminararbeit liegt der Schwerpunkt auf AdaBoost und GradientBoosting, die gemäß \textcite[S.~192]{Geron2018} zu den bekanntesten Vertretern der Boosting-Familie gehören. Die Arbeit wird die beiden Algorithmen detailliert untersuchen und verständlich darstellen.

\subsection{Motivation und Zielsetzung}

Das Ziel dieser Seminararbeit ist es, ein tiefes Verständnis für die Funktionsweise von AdaBoost und GradientBoosting zu entwickeln und deren Einsatzmöglichkeiten anhand konkreter Beispiele zu demonstrieren. Hierbei wird besonderer Wert darauf gelegt, zunächst die funktionsweise verständlich zu erläutern.

\subsection{Struktur der Arbeit}

Die Seminararbeit gliedert sich in mehrere Abschnitte, die schrittweise aufeinander aufbauen. Zunächst wird ein Überblick über die Grundlagen des maschinellen Lernens gegeben, gefolgt von einer detaillierten Betrachtung der Boosting-Algorithmen, insbesondere AdaBoost und GradientBoosting. Anschließend erfolgt ein Vergleich dieser beiden Methoden, wobei deren Stärken und Schwächen in verschiedenen Anwendungsszenarien beleuchtet werden. Abschließend wird ein Ausblick auf zukünftige Entwicklungen und mögliche Forschungsrichtungen im Bereich des Boostings gegeben.



\cite[text]{Frochte2020}
\cite[text]{SchapireFreund2012}
\cite[text]{Geron2018}
\cite[text]{James2023}
\cite[text]{Hastie2009}